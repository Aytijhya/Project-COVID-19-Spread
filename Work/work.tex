% Options for packages loaded elsewhere
\PassOptionsToPackage{unicode}{hyperref}
\PassOptionsToPackage{hyphens}{url}
%
\documentclass[
  12pt,
]{article}
\title{Work Work}
\author{Jyotishka \& his dear Aytijhya}
\date{}

\usepackage{amsmath,amssymb}
\usepackage[]{libertine}
\usepackage{iftex}
\ifPDFTeX
  \usepackage[T1]{fontenc}
  \usepackage[utf8]{inputenc}
  \usepackage{textcomp} % provide euro and other symbols
\else % if luatex or xetex
  \usepackage{unicode-math}
  \defaultfontfeatures{Scale=MatchLowercase}
  \defaultfontfeatures[\rmfamily]{Ligatures=TeX,Scale=1}
  \setmainfont[]{cochineal}
  \setsansfont[]{Linux Biolinum O}
\fi
% Use upquote if available, for straight quotes in verbatim environments
\IfFileExists{upquote.sty}{\usepackage{upquote}}{}
\IfFileExists{microtype.sty}{% use microtype if available
  \usepackage[]{microtype}
  \UseMicrotypeSet[protrusion]{basicmath} % disable protrusion for tt fonts
}{}
\makeatletter
\@ifundefined{KOMAClassName}{% if non-KOMA class
  \IfFileExists{parskip.sty}{%
    \usepackage{parskip}
  }{% else
    \setlength{\parindent}{0pt}
    \setlength{\parskip}{6pt plus 2pt minus 1pt}}
}{% if KOMA class
  \KOMAoptions{parskip=half}}
\makeatother
\usepackage{xcolor}
\IfFileExists{xurl.sty}{\usepackage{xurl}}{} % add URL line breaks if available
\IfFileExists{bookmark.sty}{\usepackage{bookmark}}{\usepackage{hyperref}}
\hypersetup{
  pdftitle={Work Work},
  pdfauthor={Jyotishka \& his dear Aytijhya},
  hidelinks,
  pdfcreator={LaTeX via pandoc}}
\urlstyle{same} % disable monospaced font for URLs
\usepackage[margin=0.9in]{geometry}
\usepackage{graphicx}
\makeatletter
\def\maxwidth{\ifdim\Gin@nat@width>\linewidth\linewidth\else\Gin@nat@width\fi}
\def\maxheight{\ifdim\Gin@nat@height>\textheight\textheight\else\Gin@nat@height\fi}
\makeatother
% Scale images if necessary, so that they will not overflow the page
% margins by default, and it is still possible to overwrite the defaults
% using explicit options in \includegraphics[width, height, ...]{}
\setkeys{Gin}{width=\maxwidth,height=\maxheight,keepaspectratio}
% Set default figure placement to htbp
\makeatletter
\def\fps@figure{htbp}
\makeatother
\setlength{\emergencystretch}{3em} % prevent overfull lines
\providecommand{\tightlist}{%
  \setlength{\itemsep}{0pt}\setlength{\parskip}{0pt}}
\setcounter{secnumdepth}{-\maxdimen} % remove section numbering
\usepackage{booktabs}
\usepackage{longtable}
\usepackage{array}
\usepackage{multirow}
\usepackage{wrapfig}
\usepackage{float}
\usepackage{colortbl}
\usepackage{pdflscape}
\usepackage{tabu}
\usepackage{threeparttable}
\usepackage{threeparttablex}
\usepackage[normalem]{ulem}
\usepackage{makecell}
\usepackage{xcolor}
\ifLuaTeX
  \usepackage{selnolig}  % disable illegal ligatures
\fi

\begin{document}
\maketitle

\definecolor{mypink1}{rgb}{0.95, 0.91, 0.85}
\definecolor{Prussian}{rgb}{0,0.2,0.4}
\definecolor{DeepBlue}{HTML}{3E0080}

\hypertarget{hello}{%
\section{HELLO}\label{hello}}

Let the vector \(X = (X_1, X_2, . . . , X_k)\) and Cov(X) = V which is
partitioned as

\[\begin{bmatrix}
V_{11} & V_{12}\\
V_{21} & V_{22}
\end{bmatrix}\]

where \(V_{11} = Cov(Y_1), V_{22} = Cov(Y_2)\) and
\(V_{12} = V_{21} = Cov(Y_1, Y_2)\), with \(Y_1 = (X_1, X_2)\),
\(Y_2 = (X_3, . . . , X_k)\). \textbackslash{} Then
\(V_{11.2} = V_{11} - V_{12}V_{22}^{-1}V_{21}\).\textbackslash{}
Consider the individual elements of \[V_{11.2} =
\begin{bmatrix}
v_{11.2} & v_{12.2} \\
v_{21.2} & v_{22.2}
\end{bmatrix}\]

We would like to compare, between \(v_{12.2}/v_{11.2}\) and
\(v_{12.2}/v_{22.2}\) and keep whichever is bigger among them and
discard the other one. This would give regression coefficient between
each pair of random variables eliminating the effect of the other
variables. This way one can select the coefficients W matrix which would
be non-zero, in the model

\[Y = W Y + \epsilon\]

and would be a way to see directional dependence in the fixed time
period. Then we would move the window. This would give a way to do path
analysis.

This process can be compared with other ones, such backward substitution
or forward selection, eliminating the multi colinearity effect.

\hypertarget{implementation}{%
\subsection{Implementation}\label{implementation}}

For each district, we started with only those districts which we found
out to be possible regressors for that district according to the
comparison rule stated above, and performed backward regression.We have
performed the computation of the estimated W matrix for 14 overlapping
timespans, viz.~

\begin{itemize}
\tightlist
\item
  Day 41 to Day 100
\item
  Day 71 to Day 130
\item
  Day 100 to Day 160
\item
  \ldots{}
\item
  \ldots{}
\item
  \ldots{}
\item
  Day 400 to Day 460
\item
  Day 430 to Day 490
\end{itemize}

\section{\textbf{\textcolor{DeepBlue}{Estimation of W in the model $ Y=WY +\epsilon $}}}
\subsection{Traditional method}

\subsection{Rigobon's Method of Partitioning}

\subsection{GKB Method}

\section{\textbf{\textcolor{DeepBlue}{Estimation reduced model $Y_t = B_1 Y_{t-1} + B_2 Y_{t-2} + \delta_t$}}}

We estimate \$ B\_1\$ , \(B_2\) using Vector Auto-regressive(2) model.

\section{\textbf{\textcolor{DeepBlue}{Estimation of the original spatio-temporal model $ Y_t = WY_t + A_1 Y_{t-1} + A_2 Y_{t-2} + \varepsilon_t $}}}

Method of estimation of W is discussed in section
1.(\textcolor{red}{Why this W and the W of the model in section 1 should be same?})\textbackslash{}
For i = 1,2; \(A_i = (I-W)B_i\)

\begin{align*}
    \text{cov}(p,q) &= \text{cov}(\alpha q + \varepsilon , \beta p + \eta)\\
    &= \alpha\beta\cdot\text{cov}(p,q) + \alpha\cdot\text{cov}(q,\eta) + \beta\cdot\text{cov}(p,\varepsilon)\\
    &= \alpha\beta\cdot\text{cov}(p,q) + \alpha\cdot\text{cov}(q,q - \beta p) + \beta\cdot\text{cov}(p,p - \alpha q)\\
    &= \alpha\cdot\text{var}(q) + \beta\cdot\text{var}(p) - \alpha\beta\cdot\text{cov}(p,q)
\end{align*}

Therefore, we have:
\[~\text{cov}(p,q) = \dfrac{1}{1+\alpha\beta}\Big[\alpha\cdot\text{var}(q) + \beta\cdot\text{var}(p)\Big] \]

Also,we calculated
\[var(q) = \frac{\beta^2 \sigma^2_{\eta,s} + \sigma^2_{\epsilon,s}}{(1-\alpha \beta)^2} \]
\[var(p) = \frac{\alpha^2 \sigma^2_{\epsilon,s} + \sigma^2_{\eta,s}}{(1-\alpha \beta)^2}\]

For \(s=1,2\); we have 6 equations using the sample estimates of the
variances and covariances over the two regimes and there are 6 unknowns
:

\(\alpha, \beta, \sigma^2_{\epsilon,1}, \sigma^2_{\epsilon,2}, \sigma^2_{\eta,1}, \sigma^2_{\eta,2}\)

\section{\textbf{\textcolor{DeepBlue}{An Approach to estimate $W$, $V_1$, and $V_2$}}}

We have been following the following spatio-temporal model:
\[Y_t = W Y_t + A_1 Y_{t-1} + A_2 Y_{t-2} + \varepsilon_t\] where
\(\varepsilon_t\) is the unknown error vector at time \(t\). Note that
in this model, \(W\), \(A_1\), and \(A_2\) are unknown \(d \times d\)
matrices.\textbackslash{}

As of now, we shall assume that the errors over time are independently
and identically distributed. We can perform the following calculation:
\[(I - W) Y_t =  A_1 Y_{t-1} + A_2 Y_{t-2} + \varepsilon_t \implies Y_t =  B_1 Y_{t-1} + B_2 Y_{t-2} + \delta_t\]

where \(B_1\), \(B_2\), \(\delta_t\) are obtained by pre-multiplying
\(A_1\), \(A_2\), \(\varepsilon_t\) by \((I - W)^{-1}\).
\textbackslash begin\{align*\}

\end{document}
